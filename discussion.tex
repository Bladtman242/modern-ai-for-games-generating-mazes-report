\section{Discussion}
%6. Discussion
%What are the strengths and shortcomings of your method? Why did you choose method X instead of Y? How well would it generalize to other game genres? How would you develop it further, if you had time?

The lattice grammars when paired with the evolutionary algorithm has a surprisingly easy time finding interesting lattices. Given that both the geno- and pheno-type are graph-like structures, we expected that we would need to solve at least some of the problems described in "Evolving neural networks through augmenting topologies"\cite{stanley:ec02}. Stanley and Miikulainen describe several problems that occur when evolving topologies (they are concerned with neural networks, whereas we work with lattices as unweighted graphs). The evolutionary algorithm used by our implementation is very basic, so we would have expected more problems, such as the algorithm simply getting stuck before finding a good solution or producing very chaotic results.

%\subsection{Blocks}

In the algorithm we have left the option to a block for each node in the lattice. A block is like a room and is itself a lattice, that describes the layout of the room. While the blocks have a huge impact on gameplay, we decided to not develop them further or include them in the overall algorithm.
The point of the project is to show of lattice-grammars and how well they work in conjunction with an evolutionary algorithm. The block level generation is not really relevant to the results, but just serve to demonstrate how to mediate some of the problems that the overall algorithm can produce. We have therefore left the specific implementation of the blocks open for extensions and future work.


%\subsection{Further work}

There are a lot of things to change or tweak on lattice grammars. They are not a full system by themselves, and, just like graph grammars, they it needs more systems to be applied in a game. For our implementation we chose to use blocks to allow for more in-depth maze generation, even if they were not used in the experiments. An extension that we did not add is node labels, which are almost essential to graph grammars to create a context for the algorithm to work with. With lattice grammars having a space to work in, that context wasn't necessary, so we left it out. Features like these leave great room for improvement to the system, and had we had more time, we could have used them to develop even more interesting mazes.


