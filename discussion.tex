\section{Discussion}
%6. Discussion
%What are the strengths and shortcomings of your method? Why did you choose method X instead of Y? How well would it generalize to other game genres? How would you develop it further, if you had time?

One surprise in this project is how easily the evolutionary algorithm finds
interesting lattices. Given that both the geno- and pheno-type are graph-like
structures, we expected that we would need to solve at least some of the
problems described in "Evolving neural networks
through augmenting topologies"\cite{stanley:ec02}. Stanley and Miikulainen
describe several problems that occur when evolving topologies (they are
concerned with neural networks, whereas we work with lattices as unweighted
graphs)

\subsection{Blocks}
In the algorithm we have left the option to a block for each node in the lattice. A block is like a room and is itself a lattice, that describes the layout of the room. While the blocks have a huge impact on gameplay, we decided to not develop them further or include them in the overall algorithm.

The point of the project is to show of lattice-grammars and how well they work in conjunction with an evolutionary algorithm. The internals of the blocks are not really relevant to the results, but just serve to mediate some of the problems that the overall algorithm can produce. We have therefore left the specific implementation of the blocks open for extensions and future work.


\subsection{Conclusion}

