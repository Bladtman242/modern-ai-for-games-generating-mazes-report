\section{Results}
The experiments yielded good results. Figure \ref{fig:maincops} shows two mazes
generated for the cops-and-robbers scenario. As intended, they both have
multiple cycles without pitfalls, they both have cul de sacs/sub-trees that can
serve as red herrings and traps, but they are not dominant. The left-most of the
mazes (generated by the "final" fitness function), has a clear repeated pattern;
a small cycle that connects to a repetition of itself. Neither the fitness
function nor any part of our method do anything to encourage repetition. It
seems seems to an inherent, though not unavoidable, property of the grammars
themselves. This is an interesting, if not entirely unexpected, result.

\begin{figure}[htbp]
\vspace{1cm}

\begin{tikzpicture}
\begin{axis}[
  xlabel=Generation,
  ylabel=fitness,
  legend pos=south east]

%cops-run plot
\addplot+[color=black,mark=*,mark options={fill=black}] table [y=best, x=G]{garun-cops-apply-limit30-100gen-properfitness};
\addlegendentry{Best fitness}

\addplot+[color=black,mark=triangle,mark options={fill=green}] table [y=med, x=G]{garun-cops-apply-limit30-100gen-properfitness};
\addlegendentry{Median fitness}

%dungeon-run plot
\addplot+[color=red,mark=*,mark options={fill=red}] table [y=best,
	x=G]{garun-dungeon-treesizeAndExpDegreeAndLogTreeCulRation-300popsize-again};
\addlegendentry{Best fitness}

\addplot+[color=red,mark=triangle,mark options={fill=red}] table [y=med,
	x=G]{garun-dungeon-treesizeAndExpDegreeAndLogTreeCulRation-300popsize-again};
\addlegendentry{Median fitness}

\end{axis}
\end{tikzpicture}

\caption{Best- and median-fitnesses for each generation of two evolutionary experiments. Black is the dungeon crawler experiment, red is the cops and robbers experiment}
\label{fig:gaplot}
\end{figure}

Figure \ref{fig:maindungeon} shows two mazes generated for the dungeon-crawler
scenario. They are both made up of large acyclic sub trees, connected by
single-node choke points. In fact, both mazes in figure \ref{fig:maindungeon}
are without cycles at all. Both mazes were found with the "final" fitness
function for dungeon crawlers. \ref{fig:otherexamples} shows another
cops-and-robber maze consisting of multiple interconnected cycles, and with a
handful of pitfalls, that has a large sub graph in the middle with high
connectivity.

It is worth noting that even though we do nothing to \textit{guarantee} that the
generated mazes are connected, that was usually the case. Our best guess is that
the fitness functions used guide the evolutionary algorithm in that direction,
because they punish pitfalls and reward tree size respectively. Figure
\ref{fig:otherexamples} shows a dungeon-crawler maze made up of multiple
disconnected graphs. This maze was found using a fitness function with terms
only for size and number of trees.

While we where looking for connected graphs, we see that the method can easily
be tweaked to find either connected or disconnected mazes.

\begin{figure}[hbtp]
	\includegraphics[height=\columnwidth]{garun-cops-apply-limit30-100gen-properfitness}
	\includegraphics[height=\columnwidth]{cops-ex3}
	\caption{Two mazes generated for the cops-and-robbers scenario}
	\label{fig:maincops}
\end{figure}


\begin{figure}[htbp]
	\includegraphics[height=\columnwidth]{garun-dungeon-treesizeAndExpDegreeAndLogTreeCulRation-300popsize-again}
	\includegraphics[height=\columnwidth]{dungeon-ex2}
	\caption{Two mazes generated for the dungeon-crawler scenario}
	\label{fig:maindungeon}
\end{figure}

\begin{figure}[htbp]
\includegraphics[height=\columnwidth,width=0.5\columnwidth]{dungeon-ex5}
\includegraphics[height=\columnwidth]{cops-ex2}
	\caption{A maze generated for the dungeon-crawler scenario (left) and
	one generated for the cops-and-robbers scenario (right)}
	\label{fig:otherexamples}
\end{figure}

As can be seen on the plot in figure \ref{fig:gaplot}, the evolutionary
algorithm converges quickly after around 40 generations. Individuals in the
final generation were often very similar or identical. The experiments in the
plot ran for 70 and 100 generation respectively, taking on the order of 1 minute
on mid-range 2014 hardware. The respective fitnesses of the two experiments
cannot be compared directly, as they use different fitness functions, but they
provide intuition about the progression of the experiments, showing that the
last \~30 generations did not change much, if at all.

The median is used, because it works with the 50\% elitist selection, showing
the tipping point between elimination and survival. Additionally, the average
was highly unstable, and often caused arithmetic overflows.

